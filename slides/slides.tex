\documentclass[11 pt]{beamer}

\usepackage[utf8]{inputenc}
\usepackage[T1]{fontenc}
\usepackage[french]{babel}

\usepackage{amsmath}
\usepackage{empheq}
\usepackage{tikz}
\usepackage{tikz-qtree}
\usepackage{listings}
\usepackage{graphicx}

\usepackage{xmpmulti}
\usepackage{animate}


% \usepackage[left=2cm,right=2cm,top=1.5cm,bottom=1.5cm]{geometry}

\title{Classification of points in a euclidean space}
\author{Luc Blassel, Romain Gautron}

\begin{document}
\maketitle

\begin{frame}{The problem}

\end{frame}

\begin{frame}[c]{A simple example}
  We have the following dataset in a $2d$ space:
  \begin{align*}
    X &=\{(1, 3),(1, 8), (2, 2), (2, 10), (3, 6), (4, 1), (5, 4), (6, 8), \\&(7, 4), (7, 7), (8, 2), (8, 5), (9, 9)\}\\
    &\\
    Y &= \{Blue,\ Blue,\ Blue,\ Blue,\ Blue,\ Blue,\ Red,\ Red,\ Red,\ Red,\ \\& Red,\ Red,\ Red \}
  \end{align*}
  We want to classify the following point: $(4,8)$\\
  What is it's color?
\end{frame}

\begin{frame}{What is being used today?}

\end{frame}

\begin{frame}{Knn}

\end{frame}

\begin{frame}{Knn}

\end{frame}

\begin{frame}{How do we partition the space?}
  Include gif of partitioning here
\end{frame}

\begin{frame}{How do we partition the space?}
  Include gif of partitioning here
\end{frame}

\begin{frame}{how do we find the k nearest neighbours?}
        % \transduration<0-95>{0}
        % \multiinclude[<+->][format=png, graphics={width=\textwidth}]{gif/cirle}
        \animategraphics[loop,controls,width=\linewidth]{10}{gif/cirle-}{0}{95}

\end{frame}



\end{document}
